% Options for packages loaded elsewhere
\PassOptionsToPackage{unicode}{hyperref}
\PassOptionsToPackage{hyphens}{url}
%
\documentclass[
]{article}
\usepackage{amsmath,amssymb}
\usepackage{lmodern}
\usepackage{ifxetex,ifluatex}
\ifnum 0\ifxetex 1\fi\ifluatex 1\fi=0 % if pdftex
  \usepackage[T1]{fontenc}
  \usepackage[utf8]{inputenc}
  \usepackage{textcomp} % provide euro and other symbols
\else % if luatex or xetex
  \usepackage{unicode-math}
  \defaultfontfeatures{Scale=MatchLowercase}
  \defaultfontfeatures[\rmfamily]{Ligatures=TeX,Scale=1}
\fi
% Use upquote if available, for straight quotes in verbatim environments
\IfFileExists{upquote.sty}{\usepackage{upquote}}{}
\IfFileExists{microtype.sty}{% use microtype if available
  \usepackage[]{microtype}
  \UseMicrotypeSet[protrusion]{basicmath} % disable protrusion for tt fonts
}{}
\makeatletter
\@ifundefined{KOMAClassName}{% if non-KOMA class
  \IfFileExists{parskip.sty}{%
    \usepackage{parskip}
  }{% else
    \setlength{\parindent}{0pt}
    \setlength{\parskip}{6pt plus 2pt minus 1pt}}
}{% if KOMA class
  \KOMAoptions{parskip=half}}
\makeatother
\usepackage{xcolor}
\IfFileExists{xurl.sty}{\usepackage{xurl}}{} % add URL line breaks if available
\IfFileExists{bookmark.sty}{\usepackage{bookmark}}{\usepackage{hyperref}}
\hypersetup{
  pdftitle={Data analysis},
  pdfauthor={Kishore Hari},
  hidelinks,
  pdfcreator={LaTeX via pandoc}}
\urlstyle{same} % disable monospaced font for URLs
\usepackage[margin=1in]{geometry}
\usepackage{color}
\usepackage{fancyvrb}
\newcommand{\VerbBar}{|}
\newcommand{\VERB}{\Verb[commandchars=\\\{\}]}
\DefineVerbatimEnvironment{Highlighting}{Verbatim}{commandchars=\\\{\}}
% Add ',fontsize=\small' for more characters per line
\usepackage{framed}
\definecolor{shadecolor}{RGB}{248,248,248}
\newenvironment{Shaded}{\begin{snugshade}}{\end{snugshade}}
\newcommand{\AlertTok}[1]{\textcolor[rgb]{0.94,0.16,0.16}{#1}}
\newcommand{\AnnotationTok}[1]{\textcolor[rgb]{0.56,0.35,0.01}{\textbf{\textit{#1}}}}
\newcommand{\AttributeTok}[1]{\textcolor[rgb]{0.77,0.63,0.00}{#1}}
\newcommand{\BaseNTok}[1]{\textcolor[rgb]{0.00,0.00,0.81}{#1}}
\newcommand{\BuiltInTok}[1]{#1}
\newcommand{\CharTok}[1]{\textcolor[rgb]{0.31,0.60,0.02}{#1}}
\newcommand{\CommentTok}[1]{\textcolor[rgb]{0.56,0.35,0.01}{\textit{#1}}}
\newcommand{\CommentVarTok}[1]{\textcolor[rgb]{0.56,0.35,0.01}{\textbf{\textit{#1}}}}
\newcommand{\ConstantTok}[1]{\textcolor[rgb]{0.00,0.00,0.00}{#1}}
\newcommand{\ControlFlowTok}[1]{\textcolor[rgb]{0.13,0.29,0.53}{\textbf{#1}}}
\newcommand{\DataTypeTok}[1]{\textcolor[rgb]{0.13,0.29,0.53}{#1}}
\newcommand{\DecValTok}[1]{\textcolor[rgb]{0.00,0.00,0.81}{#1}}
\newcommand{\DocumentationTok}[1]{\textcolor[rgb]{0.56,0.35,0.01}{\textbf{\textit{#1}}}}
\newcommand{\ErrorTok}[1]{\textcolor[rgb]{0.64,0.00,0.00}{\textbf{#1}}}
\newcommand{\ExtensionTok}[1]{#1}
\newcommand{\FloatTok}[1]{\textcolor[rgb]{0.00,0.00,0.81}{#1}}
\newcommand{\FunctionTok}[1]{\textcolor[rgb]{0.00,0.00,0.00}{#1}}
\newcommand{\ImportTok}[1]{#1}
\newcommand{\InformationTok}[1]{\textcolor[rgb]{0.56,0.35,0.01}{\textbf{\textit{#1}}}}
\newcommand{\KeywordTok}[1]{\textcolor[rgb]{0.13,0.29,0.53}{\textbf{#1}}}
\newcommand{\NormalTok}[1]{#1}
\newcommand{\OperatorTok}[1]{\textcolor[rgb]{0.81,0.36,0.00}{\textbf{#1}}}
\newcommand{\OtherTok}[1]{\textcolor[rgb]{0.56,0.35,0.01}{#1}}
\newcommand{\PreprocessorTok}[1]{\textcolor[rgb]{0.56,0.35,0.01}{\textit{#1}}}
\newcommand{\RegionMarkerTok}[1]{#1}
\newcommand{\SpecialCharTok}[1]{\textcolor[rgb]{0.00,0.00,0.00}{#1}}
\newcommand{\SpecialStringTok}[1]{\textcolor[rgb]{0.31,0.60,0.02}{#1}}
\newcommand{\StringTok}[1]{\textcolor[rgb]{0.31,0.60,0.02}{#1}}
\newcommand{\VariableTok}[1]{\textcolor[rgb]{0.00,0.00,0.00}{#1}}
\newcommand{\VerbatimStringTok}[1]{\textcolor[rgb]{0.31,0.60,0.02}{#1}}
\newcommand{\WarningTok}[1]{\textcolor[rgb]{0.56,0.35,0.01}{\textbf{\textit{#1}}}}
\usepackage{graphicx}
\makeatletter
\def\maxwidth{\ifdim\Gin@nat@width>\linewidth\linewidth\else\Gin@nat@width\fi}
\def\maxheight{\ifdim\Gin@nat@height>\textheight\textheight\else\Gin@nat@height\fi}
\makeatother
% Scale images if necessary, so that they will not overflow the page
% margins by default, and it is still possible to overwrite the defaults
% using explicit options in \includegraphics[width, height, ...]{}
\setkeys{Gin}{width=\maxwidth,height=\maxheight,keepaspectratio}
% Set default figure placement to htbp
\makeatletter
\def\fps@figure{htbp}
\makeatother
\setlength{\emergencystretch}{3em} % prevent overfull lines
\providecommand{\tightlist}{%
  \setlength{\itemsep}{0pt}\setlength{\parskip}{0pt}}
\setcounter{secnumdepth}{-\maxdimen} % remove section numbering
\ifluatex
  \usepackage{selnolig}  % disable illegal ligatures
\fi

\title{Data analysis}
\author{Kishore Hari}
\date{25/05/2022}

\begin{document}
\maketitle

\hypertarget{r-basics}{%
\subsection{R basics}\label{r-basics}}

data types: numeric, integer, character, boolean. data structures:
vectors (collection of variables of the same type), lists(collection of
variables) indexing : {[}{]} for vectors, {[}{[}{]}{]} for lists naming
: names can be assigned to elements of vectors and lists. Elements can
then be accessed via the names. For lists, one can use \$ to access
named elements. missing values: NA, dealing with missing variables
(na.rm) function: arguments - positional and named, return statement(not
necessary, but useful)

To understand and practice these elements, please install
\href{https://swirlstats.com/students.html}{``swirl''} package and
follow the instructions to install and run the beginner course
\href{https://github.com/swirldev/swirl_courses/blob/master/README.md}{at
this link}.

\hypertarget{dataframes}{%
\subsection{Dataframes}\label{dataframes}}

Dataframe is the data structure used to store and work with tabular
data. rows (observations), columns(variables) dataframes are lists of
columns, so list-like indexing can be done.

\hypertarget{tidyverse}{%
\subsection{tidyverse}\label{tidyverse}}

Tidyverse is a set of packages designed to wrangle, analyze and
visualize tabular data. Here, We will see functions that can be used to
wrangle and modify dataframes. A file called dataSet.dat has been
included in the repository. The table in the file is referred to as
count matrix.

\begin{enumerate}
\def\labelenumi{\arabic{enumi}.}
\tightlist
\item
  Take the count matrix, classify genes as high or low expression in a
  given sample using ensemble mean as the reference.
\end{enumerate}

\begin{Shaded}
\begin{Highlighting}[]
\FunctionTok{library}\NormalTok{(tidyverse)}
\end{Highlighting}
\end{Shaded}

\begin{verbatim}
## Warning: package 'tidyverse' was built under R version 4.1.3
\end{verbatim}

\begin{verbatim}
## -- Attaching packages --------------------------------------- tidyverse 1.3.1 --
\end{verbatim}

\begin{verbatim}
## v ggplot2 3.3.5     v purrr   0.3.4
## v tibble  3.1.6     v dplyr   1.0.8
## v tidyr   1.2.0     v stringr 1.4.0
## v readr   2.1.2     v forcats 0.5.1
\end{verbatim}

\begin{verbatim}
## Warning: package 'ggplot2' was built under R version 4.1.3
\end{verbatim}

\begin{verbatim}
## Warning: package 'tibble' was built under R version 4.1.3
\end{verbatim}

\begin{verbatim}
## Warning: package 'tidyr' was built under R version 4.1.3
\end{verbatim}

\begin{verbatim}
## Warning: package 'readr' was built under R version 4.1.3
\end{verbatim}

\begin{verbatim}
## Warning: package 'dplyr' was built under R version 4.1.3
\end{verbatim}

\begin{verbatim}
## Warning: package 'stringr' was built under R version 4.1.3
\end{verbatim}

\begin{verbatim}
## Warning: package 'forcats' was built under R version 4.1.3
\end{verbatim}

\begin{verbatim}
## -- Conflicts ------------------------------------------ tidyverse_conflicts() --
## x dplyr::filter() masks stats::filter()
## x dplyr::lag()    masks stats::lag()
\end{verbatim}

\begin{enumerate}
\def\labelenumi{\roman{enumi})}
\tightlist
\item
  read the file
\end{enumerate}

\begin{Shaded}
\begin{Highlighting}[]
\NormalTok{df }\OtherTok{\textless{}{-}} \FunctionTok{read\_delim}\NormalTok{(}\StringTok{"dataSet.dat"}\NormalTok{, }\AttributeTok{delim =} \StringTok{" "}\NormalTok{, }\AttributeTok{col\_names =}\NormalTok{ F)}
\end{Highlighting}
\end{Shaded}

\begin{verbatim}
## Rows: 3995 Columns: 15
## -- Column specification --------------------------------------------------------
## Delimiter: " "
## dbl (15): X1, X2, X3, X4, X5, X6, X7, X8, X9, X10, X11, X12, X13, X14, X15
## 
## i Use `spec()` to retrieve the full column specification for this data.
## i Specify the column types or set `show_col_types = FALSE` to quiet this message.
\end{verbatim}

\begin{Shaded}
\begin{Highlighting}[]
\FunctionTok{colnames}\NormalTok{(df) }\OtherTok{\textless{}{-}} \FunctionTok{c}\NormalTok{(}\StringTok{"SampleID"}\NormalTok{, }\FunctionTok{paste0}\NormalTok{(}\StringTok{"Gene"}\NormalTok{, }\DecValTok{1}\SpecialCharTok{:}\DecValTok{14}\NormalTok{))}
\NormalTok{means }\OtherTok{\textless{}{-}} \FunctionTok{sapply}\NormalTok{(df, mean, }\AttributeTok{na.rm =}\NormalTok{ T)}
\NormalTok{means}
\end{Highlighting}
\end{Shaded}

\begin{verbatim}
##      SampleID         Gene1         Gene2         Gene3         Gene4 
##  4.954774e+04 -3.323243e-02 -1.346634e-01 -1.432902e+00 -2.523484e+00 
##         Gene5         Gene6         Gene7         Gene8         Gene9 
## -1.802171e+00  1.608014e+00  3.973519e+00  1.908778e+00  1.313205e+00 
##        Gene10        Gene11        Gene12        Gene13        Gene14 
##  1.823515e+00  2.236143e+00  2.331142e+00  2.198053e+00  2.210632e+00
\end{verbatim}

\begin{Shaded}
\begin{Highlighting}[]
\FunctionTok{sapply}\NormalTok{(df, class)}
\end{Highlighting}
\end{Shaded}

\begin{verbatim}
##  SampleID     Gene1     Gene2     Gene3     Gene4     Gene5     Gene6     Gene7 
## "numeric" "numeric" "numeric" "numeric" "numeric" "numeric" "numeric" "numeric" 
##     Gene8     Gene9    Gene10    Gene11    Gene12    Gene13    Gene14 
## "numeric" "numeric" "numeric" "numeric" "numeric" "numeric" "numeric"
\end{verbatim}

comparision

\begin{Shaded}
\begin{Highlighting}[]
\NormalTok{fn }\OtherTok{\textless{}{-}} \ControlFlowTok{function}\NormalTok{(x) \{}
    \FunctionTok{ifelse}\NormalTok{(x }\SpecialCharTok{\textgreater{}} \FunctionTok{mean}\NormalTok{(x), }\StringTok{"1"}\NormalTok{, }\StringTok{"0"}\NormalTok{)}
\NormalTok{\}}
\NormalTok{df1 }\OtherTok{\textless{}{-}} \FunctionTok{data.frame}\NormalTok{(}\FunctionTok{sapply}\NormalTok{(df[, }\DecValTok{2}\SpecialCharTok{:}\DecValTok{15}\NormalTok{], fn))}
\end{Highlighting}
\end{Shaded}

Repeating same operations using dplyr/tidyverse verbs.

\begin{Shaded}
\begin{Highlighting}[]
\NormalTok{df2 }\OtherTok{\textless{}{-}} \FunctionTok{read\_delim}\NormalTok{(}\StringTok{"dataSet.dat"}\NormalTok{, }\AttributeTok{delim =} \StringTok{" "}\NormalTok{, }\AttributeTok{col\_names =}\NormalTok{ F) }\SpecialCharTok{\%\textgreater{}\%}
    \FunctionTok{set\_names}\NormalTok{(}\FunctionTok{c}\NormalTok{(}\StringTok{"SampleID"}\NormalTok{, }\FunctionTok{paste0}\NormalTok{(}\StringTok{"Gene"}\NormalTok{, }\DecValTok{1}\SpecialCharTok{:}\DecValTok{14}\NormalTok{))) }\SpecialCharTok{\%\textgreater{}\%}
    \FunctionTok{mutate}\NormalTok{(}\FunctionTok{across}\NormalTok{(}\AttributeTok{.cols =} \FunctionTok{contains}\NormalTok{(}\StringTok{"Gene"}\NormalTok{), }\AttributeTok{.fns =}\NormalTok{ fn))}
\end{Highlighting}
\end{Shaded}

\begin{verbatim}
## Rows: 3995 Columns: 15
## -- Column specification --------------------------------------------------------
## Delimiter: " "
## dbl (15): X1, X2, X3, X4, X5, X6, X7, X8, X9, X10, X11, X12, X13, X14, X15
## 
## i Use `spec()` to retrieve the full column specification for this data.
## i Specify the column types or set `show_col_types = FALSE` to quiet this message.
\end{verbatim}

\begin{enumerate}
\def\labelenumi{\arabic{enumi}.}
\setcounter{enumi}{1}
\tightlist
\item
  arrange the dataframe by Genes1, 2, 3
\end{enumerate}

\begin{Shaded}
\begin{Highlighting}[]
\NormalTok{df3 }\OtherTok{\textless{}{-}}\NormalTok{ df2 }\SpecialCharTok{\%\textgreater{}\%} \FunctionTok{arrange}\NormalTok{(Gene1, Gene2, Gene3)}
\end{Highlighting}
\end{Shaded}

\begin{enumerate}
\def\labelenumi{\arabic{enumi}.}
\setcounter{enumi}{2}
\tightlist
\item
  select even numbered genes
\end{enumerate}

\begin{Shaded}
\begin{Highlighting}[]
\NormalTok{naFunc }\OtherTok{\textless{}{-}} \ControlFlowTok{function}\NormalTok{(x) \{}
    \ControlFlowTok{if}\NormalTok{ (}\FunctionTok{all}\NormalTok{(}\FunctionTok{is.na}\NormalTok{(x)))}
\NormalTok{        T}
    \ControlFlowTok{else}
\NormalTok{        F}
\NormalTok{\}}
\NormalTok{df3 }\OtherTok{\textless{}{-}}\NormalTok{ df2 }\SpecialCharTok{\%\textgreater{}\%} \FunctionTok{select}\NormalTok{(}\FunctionTok{paste0}\NormalTok{(}\StringTok{"Gene"}\NormalTok{, (}\DecValTok{1}\SpecialCharTok{:}\DecValTok{7}\NormalTok{)}\SpecialCharTok{*}\DecValTok{2}\NormalTok{))}
\end{Highlighting}
\end{Shaded}

\begin{enumerate}
\def\labelenumi{\arabic{enumi}.}
\setcounter{enumi}{3}
\tightlist
\item
  filter the samples that have Gene1 and Gene2 are overexpressed and
  Gene3 is downregulated.
\end{enumerate}

\begin{Shaded}
\begin{Highlighting}[]
\NormalTok{df3 }\OtherTok{\textless{}{-}}\NormalTok{ df }\SpecialCharTok{\%\textgreater{}\%} \FunctionTok{filter}\NormalTok{(Gene1 }\SpecialCharTok{==} \DecValTok{1} \SpecialCharTok{\&}\NormalTok{ Gene2 }\SpecialCharTok{==} \DecValTok{1} \SpecialCharTok{\&}\NormalTok{ Gene3 }\SpecialCharTok{==}\DecValTok{0}\NormalTok{)}
\end{Highlighting}
\end{Shaded}

\begin{enumerate}
\def\labelenumi{\arabic{enumi}.}
\setcounter{enumi}{4}
\tightlist
\item
  Find the samples that have overexpression of Gene4 or overexpression
  of both Gene3 and Gene5, and display the levels of Gene1 and Gene2 for
  these samples in ascending order of Gene1
\end{enumerate}

\begin{Shaded}
\begin{Highlighting}[]
\NormalTok{df3 }\OtherTok{\textless{}{-}}\NormalTok{ df2 }\SpecialCharTok{\%\textgreater{}\%} \FunctionTok{filter}\NormalTok{(Gene4 }\SpecialCharTok{==} \StringTok{"1"} \SpecialCharTok{|}\NormalTok{ (Gene3 }\SpecialCharTok{==} \StringTok{"1"} \SpecialCharTok{\&}\NormalTok{ Gene5 }\SpecialCharTok{==} \StringTok{"1"}\NormalTok{)) }\SpecialCharTok{\%\textgreater{}\%}
    \FunctionTok{select}\NormalTok{(Gene1, Gene2) }\SpecialCharTok{\%\textgreater{}\%} \FunctionTok{arrange}\NormalTok{(Gene1)}
\end{Highlighting}
\end{Shaded}

\begin{enumerate}
\def\labelenumi{\arabic{enumi}.}
\setcounter{enumi}{5}
\tightlist
\item
  Gene1,2,3 belong to the same pathway A. So, get the average level of
  expression of the other genes in samples that have low, high and
  medium levels of expression of pathway A.
\end{enumerate}

\hypertarget{gather-and-spread}{%
\subsubsection{Gather and spread}\label{gather-and-spread}}

Gather and spread are functions that are used to

\begin{Shaded}
\begin{Highlighting}[]
\NormalTok{df4 }\OtherTok{\textless{}{-}}\NormalTok{ df2 }\SpecialCharTok{\%\textgreater{}\%} \FunctionTok{gather}\NormalTok{(}\AttributeTok{key =}  \StringTok{"Gene"}\NormalTok{, }\AttributeTok{value =} \StringTok{"Expression"}\NormalTok{, }\SpecialCharTok{{-}}\NormalTok{SampleID)}
\NormalTok{df5 }\OtherTok{\textless{}{-}}\NormalTok{ df4 }\SpecialCharTok{\%\textgreater{}\%} \FunctionTok{spread}\NormalTok{(}\AttributeTok{key =}\NormalTok{ Gene, }\AttributeTok{value =}\NormalTok{ Expression)}
\end{Highlighting}
\end{Shaded}


\end{document}
